% Awesome Source CV LaTeX Template
%
% This template has been downloaded from:
% https://github.com/darwiin/awesome-neue-latex-cv
%
% Author:
% Christophe Roger
%
% Template license:
% CC BY-SA 4.0 (https://creativecommons.org/licenses/by-sa/4.0/)

%Section: Project at the top
\sectionTitle{Projects}{\faWrench}
%\renewcommand{\labelitemi}{$\bullet$}

\begin{projects}

    \project
        {January 2017}
        {Conferences' atendees live tracking and analysis}
        {IBM}
        {October 2017}
        {
            \begin{itemize}
                \item In less than a month, our team built a proof of concept 
                    of a solution to the registration of attendees in 
                    conferences, while also providing mobile applications and 
                    an RFID tracking system used to identify the attendees' 
                    participation and movement around the event's venue.
                \item Used the captured data to gain insights and create a 
                    dashboard using the Watson Analytics platform.
                \item Created reports by cleaning and processing the raw data 
                    using Pandas and plotting with Matplotlib.
                \item A fully-working demo was presented to IBM's global 
                    leadership during the \textit{Interconnect} conference in 
                    Las Vegas.
            \end{itemize}
        }
        {
            Watson Analytics,
            Python,
            Pandas,
            Matplotlib,
            MySQL,
            ETL,
            RFID, 
            Linux,
            CSV,
            SQLite
        }

    \emptySeparator

    \project
        {June 2016}
        {Cognitive Concierge}
        {IBM}
        {October 2016}
        {
            \begin{itemize}
                \item By using IBM's Watson, our team trained and configured 
                    speech recognition patterns along with questions and 
                    answers in order to program a set of humanoid robots who 
                    could understand and answer natural language questions 
                    about the conference.
                \item A fully-working demo was presented publicly during the 
                    \textit{World of Watson} conference in Las Vegas.
            \end{itemize}
        }
        {
            SoftBank Robotics' Nao and Pepper humanoid robots,
            Watson,
            Natural-language processing,
            Speech recognition,
            Text-to-speech,
            Speech-to-text,
            Linux,
            MySQL,
            Bluemix
        }

    \emptySeparator

    \project
        {January 2016}
        {Hadoop Raspberry Pi Cluster}
        {IBM}
        {March 2016}
        {
            \begin{itemize}
                \item Built a fully-working 12-node Hadoop cluster with 
                    Raspberry Pies including setting up the environment in each 
                    node, testing it and presenting a demo to high management 
                    in order to demonstrate the feasibility of using 
                    Raspberries as an affordable cloud offering to entry-level 
                    clients.
                \item Documented the entire process and published it on the 
                    \href{https://developer.ibm.com/recipes/tutorials/building-a-hadoop-cluster-with-raspberry-pi/}
                    {\underline{IBM developerWorks site}}.
            \end{itemize}
        }
        {
            Hadoop,
            HDFS,
            Raspbian,
            Raspberry Pi,
            SSH
        }

    \emptySeparator

    \project
        {August 2016}
        {Travel searcher}
        {Independent}
        {April 2018}
        {
            \begin{itemize}
                \item By analyzing Google's travel data, built a wrapper and 
                    notification system to inform me of affordable flights for 
                    destinations, prices and schedules I decided.
                \item Published all the code and documentation in 
                    \href{https://github.com/alanverdugo/QPX}{\underline{my 
                    github repository}}.
            \end{itemize}
        }
        {
            Python,
            Google QPX API,
            Linux,
            Cron,
            Git
        }

\end{projects}
